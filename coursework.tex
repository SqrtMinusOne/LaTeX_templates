\documentclass[a4paper, 14pt]{extarticle}


% ====== Localization & fonts ======
\usepackage{xcolor}
\usepackage{fontspec}
\usepackage{polyglossia} % Russian lang
\setdefaultlanguage{russian}
\setmainfont{Times New Roman}[Mapping=tex-text]
\newfontfamily\cyrillicfont{Times New Roman}[Mapping=tex-text]

% TODO look for better; don't like this one
\setmonofont[Scale=0.7]{DejaVu Sans Mono}
\newfontfamily{\cyrillicfonttt}{DejaVu Sans Mono}[Scale=0.7]

% ====== Math ======
\usepackage{amsmath} % Math stuff
\usepackage{gensymb}
\usepackage{amssymb}

% ====== Other text spacing ======

\usepackage{setspace} % String spacing
\onehalfspacing{}

% \usepackage{parskip} % The red line
\setlength{\parskip}{6pt} % Sep beetween paragraphs
\setlength{\parindent}{18pt}
\usepackage{indentfirst}

% ====== Source code listings ======

\usepackage{listings} % Source code
\lstset{
    basicstyle=\ttfamily,
    breaklines=true,
    breakatwhitespace=true,
    commentstyle=\color{green},
    keywordstyle=\bfseries\color{blue},
    tabsize=4
}

\lstdefinestyle{num}{
    numbers=left,
    numbersep=0.5cm,
    numberstyle=\ttfamily,
    xleftmargin=0.85cm,
}

\lstdefinestyle{framed_num}{
    frame=single,
    numbers=left,
    numbersep=0.5cm,
    numberstyle=\ttfamily\color{red},
    xleftmargin=1cm,
    framexleftmargin=1cm
}

\lstdefinestyle{sql_style}{
    morekeywords=[10]{while, do, procedure, begin, end, if, tinyint, enum, datetime, boolean, declare, function, return, deterministic, is}
}

\lstdefinestyle{glsl_style}{
    morekeywords=[10]{version, layout, location, in, out, vec3, uniform, vec4, normalize, cross, dot, precision, bool}
}

\renewcommand{\lstlistingname}{Листинг}

% ====== CSV ======
\usepackage{csvsimple} % Load CSV

% ====== References ======
\usepackage{hyperref} % Links
\hypersetup{
    colorlinks,
    citecolor=black,
    filecolor=black,
    linkcolor=black,
    urlcolor=black
}

% ====== TOC ======
\usepackage{tocloft}
\setlength{\cftbeforesecskip}{3pt}

\renewcommand{\cftsecfont}{\cyrillicfont}
\renewcommand{\cftsubsecfont}{\cyrillicfont}

% ====== Captions ======
\usepackage{caption}
\usepackage{subcaption}
\captionsetup[figure]{name=Рисунок, labelsep=endash, parskip=0pt}

\captionsetup[lstlisting]{
    labelsep=period,
    justification=RaggedLeft,
    parskip=0pt,
    singlelinecheck=false,
    skip=3pt
}

\captionsetup[table]{
    labelsep=period,
    justification=RaggedLeft,
    parskip=0pt,
    singlelinecheck=false,
    skip=3pt
}

\captionsetup[figure]{
    name=Рисунок,
    justification=centering,
    labelsep=period,
    parskip=6pt,
    skip=3pt
}

% ====== Misc ======
\usepackage{metalogo} % Logos for LaTeX
\usepackage{lipsum} % Lorem ipsum
\usepackage{graphicx} % Pictures
\usepackage{tabularx} % X-tables
\usepackage{csquotes} % French quoutes
\usepackage{multirow}

\usepackage{lastpage}

\usepackage{placeins} % FloatBarrier

\usepackage[final]{pdfpages} % Include PDF
\setboolean{@twoside}{false}


% ====== Plotting ======
\usepackage{tikz}
\usepackage{pgfplots}
\usepackage{pgfplotstable}
\pgfplotsset{compat=newest}
\usepgfplotslibrary{dateplot}


% ====== My commands ======
\newcommand{\addonsubheader}[1]{{\center{\bfseries \vspace{-0.5cm} #1} \\ \vspace{15pt}}}

\newcommand{\screenshot}[3]{
    \begin{figure}[h]
        \centering
        \includegraphics[#1]{#2}
        \caption{#3}
    \end{figure}
}

% ====== Counters ======

\numberwithin{equation}{section}

\newcounter{stepcounter}[subsubsection]

\newcommand\steppar[1][\DefaultOpt]{
    \def\DefaultOpt{#1}
    \paragraph{Шаг №\thestepcounter.}
    \refstepcounter{stepcounter}
}

% ====== Document sectioning ======
\usepackage{titlesec}
\titleformat{\section}{\filcenter\bfseries}{\thesection. }{0pt}{\MakeUppercase}
\titleformat*{\subsection}{\bfseries}
\titleformat*{\subsubsection}{\bfseries}
\titleformat*{\paragraph}{\bfseries}
\titleformat*{\subparagraph}{\bfseries\itshape} % chktex 6

\titlespacing*{\subsection}{0pt}{24pt}{3pt}
\titlespacing*{\subsubsection}{0pt}{15pt}{0pt}
\titlespacing*{\paragraph}{0pt}{15pt}{6pt}
\titlespacing*{\subparagraph}{0pt}{15pt}{3pt}

\newcommand\sectionbreak{\clearpage}
% ====== Page layout ======
\usepackage[ % Margins
left=3cm,
right=2cm,
top=2cm,
bottom=2cm
]{geometry}


% ====== Itemize ======
\usepackage{enumitem}
\setlist{nosep, topsep=-10pt, leftmargin=*} % Remove sep-s beetween list elements
\setlist[itemize, 2]{label = {\scriptsize\ensuremath{\blacksquare}}}
\setlist[itemize, 3]{label = {---}}

\begin{document}
\begin{titlepage}
    \centering
    {\bfseries
        \uppercase{
            Минобрнауки России \\
            Санкт-Петербургский государственный \\
            Электротехнический университет \\
            \enquote{ЛЭТИ} им. В.И.Ульянова (Ленина)\\
        }
        Кафедра МО ЭВМ

        \vspace{\fill}
        \uppercase{Курсовая работа} \\
        по дисциплине \enquote{Компьютерная графика} \\
        Тема: Реализация сцены с визуализацией цветового распределения 3Д
    }

    \vspace{\fill}
    \begin{tabularx}{0.8\textwidth}{l X c r}
        Студенты гр. 6304 & & \underline{\hspace{3cm}} & Корытов П.В.\\
                          & & \underline{\hspace{3cm}} & Пискунов Я.А.\\
        Преподаватель & & \underline{\hspace{3cm}} & Герасимова Т.В.
    \end{tabularx}

    \vspace{1cm}
    Санкт-Петербург \\
    \the\year{}
\end{titlepage}

\newpage

\tableofcontents

\newpage
\section{Постановка задачи}\label{sec:task}

\section{Выводы}

\begin{thebibliography}{9} 
    \addcontentsline{toc}{section}{Список литературы}
    \bibitem{srtmdata} SRTM Data – CGIAR-CSI SRTM [Электронный ресурс] --- Режим доступа: http://srtm.csi.cgiar.org/srtmdata/, владелец --- GoDaddy.com, LLC.\@
    \bibitem{phong} learnopengl. Урок 2.2 — Основы освещения [Электронный ресурс] --- Режим доступа: https://habr.com/ru/post/333932/, автор --- 0xEEd, владелец --- OOO ``Хабр''
    \bibitem{python} The Python Language Reference [Электронный ресурс] --- Режим доступа: https://docs.python.org/3/reference/, владелец --- Python Software Foundation
    \bibitem{georasters} Georasters Documentation [Электронный ресурс] --- Режим доступа: https://georasters.readthedocs.io/en/latest/, владелец --- Ömer Özak
 \end{thebibliography}

\section*{Приложение А}\label{add:py:geotiff}
\addcontentsline{toc}{section}{Приложение А. Содержимое файла geotiff\_processor.py}
\addonsubheader{Содержимое файла geotiff\_processor.py}

\end{document}
